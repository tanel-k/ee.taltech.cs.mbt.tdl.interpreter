\documentclass[12pt,oneside,a4paper,draft,notitlepage]{article}
\usepackage[backend=biber]{biblatex}
\usepackage{setspace}

\title{
	Implementation and Validation of an Interpreter
	for the Test Scenario Specification Language TDL\textsuperscript{TP}\\
}
\author{
	Student: Tanel Prikk\\
	\texttt{tanel.prikk@taltech.ee}
	\and
	Instructor: Jüri Vain\\
	\texttt{juri.vain@taltech.ee}
}
\date{March 2019}

\addbibresource{Sources.bib}
\pagenumbering{gobble}
\singlespacing

\begin{document}
	\maketitle
	\section*{Background}
	\begin{itemize}
		\item What is MBT?
		\item Who uses MBT?
		\item What is test description?
		\item TDL and its limitations.
		\item What is Uppaal?
		\item TDL(TP)'s advantages.
		\item What is TDL?
		\item Goal. The objective of this project is to develop an interpreter for the TDL\textsuperscript{TP} test specification language. 
		\item Where will this thing find use? (Cyber-phys etc.)
	\end{itemize}

	\section*{Expected Results}
	\par The interpreter program produced as the objective of this paper will accept as input:
	\begin{itemize}
		\item a UTA system model M\textsuperscript{SUT} (in XML); as well as
		\item an expression $\phi^{TP}$ in the TDL\textsuperscript{TP} language.
	\end{itemize}

	\par Based on the provided inputs, the interpreter produces as output a UTA test model, serialized into XML. This test model must implement the test specification represented by $\phi^{TP}$ and must be compatible with the UTA tool.

	\bigskip

	\par The synthesis of a test model M based on M\textsuperscript{SUT} and $\phi^{TP}$ shall include the following:
	\begin{itemize}
		\item the simplification of the TDL\textsuperscript{TP} expression based on the system model and proven simplification rules for the language;
		\item the creation of a version of the original system model M\textsuperscript{SUT} modified with flag structures based on the simplified TDL\textsuperscript{TP} expression (M\textsuperscript{T});
		\item the creation of an expression model derived from the sub-formulas in the TDL\textsuperscript{TP} expression (M\textsuperscript{E});
		\item the creation of a stop-watch model for encoding test verdict information (M\textsuperscript{SW}); and
		\item the parallel composition of the above-mentioned three models, \\
		$M^{T} \vert M^{E} \vert M^{SW}$ in order to produce the final test model.
	\end{itemize}

	\par The resulting test model can be used in the Uppaal tool to validate the original model based on the test specification. The model shall include diagnostic information encoded as a Boolean array that mirrors the parse tree of $\phi^{TP}$, which the user will find useful when analyzing failure cases for the system.

	\section*{Validation}
	\par The interpreter is a prototype so there will be no direct efficiency requirements imposed on it in the context of this project. However, it must be ensured that the test models generated by the interpreter are functionally correct.

	\bigskip

	\par In order to validate whether the interpreter produces functionally correct models, we plan to present timed computation tree logic (TCTL) constraints equivalent to TDL\textsuperscript{TP} expressions in Uppaal's query language. By comparing the test sequences generated by these we can check the functional correctness of the TDL\textsuperscript{TP} interpreter. (\textit{I am not sure this will make sense to a fresh pair of eyes. Do we plan to validate on a few examples? Perhaps we can prove the validity of a few examples? Can we have an algorithmic proof of validity?})

	\section*{Methodology}
	\par When implementing the interpreter, we shall take a bottom-up approach. In basic terms, this means we will develop input/output serialization components (such as the TDL\textsuperscript{TP} syntax parser and the UTA model serialization component) prior to moving on to the object models that will be used in the primary deliverable. The programming language of choice for core components is Java.

	\bigskip

	\par For the TDL\textsuperscript{TP} syntax parser, we will decide on a suitable Java-based parser generator which produces artifacts that we can easily map to an implementation-independent object model of our own using an adapter. We will take this approach to save time on building a custom parser while leaving the option to simply swap out the existing parser without affecting the top-level dependent artifacts.

	\bigskip

	\par As for the model serialization component, we will leverage the existing \texttt{libutap} parser library \cite{libutapsite} developed for Uppaal and provided under the LGPL license. The library is implemented in \texttt{C++}, so an adapter needs to be written in order to accommodate it in the artifact tree.

	\bigskip

	\par Additional research and analysis will be required to determine a suitable internal data structure which will facilitate the simplification of TDL\textsuperscript{TP} expressions based on a set of existing rules.

	\bigskip

	\par \textit{Do we need anything else here?}

	\printbibliography[
		title=Sources
	]

\end{document}

