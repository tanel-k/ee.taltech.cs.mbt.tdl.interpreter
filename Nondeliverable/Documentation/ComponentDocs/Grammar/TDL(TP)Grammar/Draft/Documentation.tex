\documentclass[12pt,oneside,a4paper,notitlepage]{report}

\usepackage[backend=biber,sorting=none,maxbibnames=99]{biblatex}
\usepackage{setspace}
\usepackage{syntax}
\usepackage{wasysym}
\usepackage{textcomp}
\usepackage{newfloat}
\usepackage{multirow}
\usepackage{listings}

\title{
	Generation of a Parser for the TDL\textsuperscript{TP} Expression Language
}
\author{Tanel Prikk}
\DeclareFloatingEnvironment[
	fileext   = logr,
	listname  = {List of Grammars},
	name      = Grammar,
	placement = htp
]{GrammarWrapper}
\setlength{\grammarindent}{5em}
\setlength{\grammarparsep}{5pt plus 1pt minus 1pt}

\addbibresource{Sources.bib}
\pagenumbering{gobble}
\singlespacing

\newcommand{\texttilde}{\raisebox{0.5ex}{\texttildelow}}

\begin{document}
	\maketitle

	\section*{Background}
	\par Section 3 in \cite{tdlarticle} describes the syntax of the TDL\textsuperscript{TP} expression language in BNF:

	\begin{GrammarWrapper}
		\begin{grammar}
		% https://tex.stackexchange.com/questions/24886/which-package-can-be-used-to-write-bnf-grammars
		% http://texdoc.net/texmf-dist/doc/latex/mdwtools/syntax.pdf
		<Expression>	::=	'(' <Expression> ')'
		\alt 				'A' <TrapsetExpression>
		\alt 				'E' <TrapsetExpression>
		\alt 				<UnaryOp> <Expression>
		\alt 				<Expression> <BinaryOp> <Expression>
		\alt 				<Expression> '\texttilde\textgreater' <Expression>
		\alt 				<Expression> '\texttilde\textgreater' '[' <RelOp> <NUM> ']' <Expression>
		\alt 				'\#' <Expression> <RelOp> <NUM>
		
		<TrapsetExpression>	::=	'(' <TrapsetExpression> ')'
		\alt						'!' <ID>
		\alt 						<ID> '\textbackslash' <ID>
		\alt						<ID> ';' <ID>
		
		<UnaryOp>	::= 'not'
		
		<BinaryOp>	::= '\&' | 'or' | '=\textgreater' | '\textless=\textgreater'
		
		<RelOp> 	::= '\textless' | '=' | '\textgreater' | '\textless=' | '\textgreater='
		
		<ID> 		::= 'TR' <NUM>
		
		<NUM> 		::= ('0' ... '9')+
		\end{grammar}
		\caption{TDL\textsuperscript{TP} grammar}\label{bnf:original}
	\end{GrammarWrapper}

	\newpage

	As part of the basic functionality of the interpreter, we need to be able to parse an input expression in TDL\textsuperscript{TP} and traverse the resulting concrete syntax tree (i.e parse tree). The first step towards achieving this functionality is the generation of a lexer and a parser for the language. 

	\section*{Requirements Analysis}
	% https://arxiv.org/ftp/arxiv/papers/1409/1409.2378.pdf ?
	% https://nm.wu.ac.at/nm/strembeck/publications/europlop09.pdf ?
	\par The BNF in grammar~\ref{bnf:original} can be adjusted to make it more convenient for the user to specify expressions. We list the adjustments below:
	\begin{itemize}
		\item to reduce ambiguity, logical operators such as implication (\texttt{'=\textgreater'}) and iff (\texttt{'\textless=\textgreater'}) should be differentiated from relational operators over natural numbers which also use the \texttt{'='} token (i.e \texttt{'\textgreater='}, \texttt{'\textless='}) - we achieve this by replacing \texttt{'='} with the equally easy-to-use token \texttt{'-'} in the logical operators: \texttt{'-\textgreater'}, \texttt{'\textless-\textgreater'};
		\item if at some point we wish to have \texttt{'\textless ID\textgreater'} values that can start with either \texttt{'A'} or \texttt{'E'}, it will be visually difficult to identify usage of the universal (\texttt{'A'}) and the existential (\texttt{'E'}) quantifier in an expression - to fix this, we must enforce the use of brackets around the arguments of these quantifiers;
		\item for consistency, logical negation (\texttt{'not'}), conjunction (\texttt{'\&'}), disjunction (\texttt{'or'}), implication (\texttt{'-\textgreater'}) and equivalence (\texttt{'\textless-\textgreater'}) should all be either textual (\texttt{'not'}, \texttt{'and'}, \texttt{'or'}, \texttt{'implies'}, \texttt{'iff'}) or symbolic (\texttt{'\texttilde'}, \texttt{'\&'}, \texttt{'|'}, \texttt{'-\textgreater'}, \texttt{'\textless-\textgreater'}) - there are arguments for both approaches but the conciseness of the symbolic approach is a good enough argument for it to be chosen;
		\item for consistency, the time-bounded leads-to operator \\ (\texttt{'\texttilde\textgreater [\textless RelOp\textgreater \textless NUM\textgreater]'}) and the conditional repetition operator \\ (\texttt{'\#~\textless Expression\textgreater \textless RelOp\textgreater \textless NUM\textgreater'}) should use the same wrapper construct for the bound relation (\texttt{\textless RelOp\textgreater \textless NUM\textgreater}) - square brackets;
		\item for multi-token operators (e.g \texttt{'-\textgreater'}, \texttt{'\textless-\textgreater'}, \texttt{'\textless='}, \texttt{'\textgreater='}), we should allow an arbitrary amount of whitespace between component tokens to reduce the number of potential typing errors.
	\end{itemize}

	\newpage

	\par A new grammar can be produced based on grammar~\ref{bnf:original} and the adjustments we specified previously. This grammar is displayed in the listing below. Each distinguishable expression type is also provided with an identifying label in brackets.
		
	\begin{GrammarWrapper}
		\begin{grammar}
			% https://tex.stackexchange.com/questions/24886/which-package-can-be-used-to-write-bnf-grammars
			% http://texdoc.net/texmf-dist/doc/latex/mdwtools/syntax.pdf
			<Expression>	::=	'(' <Expression> ')'
			\alt 				'A' '(' <TrapsetExpression> ')' \textit{(UQTE)}
			\alt 				'E' '(' <TrapsetExpression> ')' \textit{(EQTE)}
			\alt 				<UnaryOp> <Expression> \textit{(NE)}
			\alt 				<Expression> <BinaryOp> <Expression> \textit{(BLE)}
			\alt 				<Expression> '\texttilde' '\textgreater' <Expression> \textit{(LTE)}
			\alt 				<Expression> '\texttilde' '\textgreater' '[' <RelOp> <NUM> ']' <Expression> \textit{(BLTE)}
			\alt 				'\#' <Expression> '[' <RelOp> <NUM> ']' \textit{(CRE)}
	
			<TrapsetExpression>	::=	'(' <TrapsetExpression> ')'
			\alt						'!' <ID> \textit{(CTE)}
			\alt 						<ID> '\textbackslash' <ID> \textit{(RCTE)}
			\alt						<ID> ';' <ID> \textit{(LPTE)}
	
			<UnaryOp>	::= '\texttilde'
	
			<BinaryOp>	::= '\&' | '|' | '-' '\textgreater' | '\textless' '-' '\textgreater'
	
			<RelOp> 	::= '\textless' | '=' | '\textgreater' | '\textless' '=' | '\textgreater' '='
	
			<ID> 		::= 'TR' <NUM>
	
			<NUM> 		::= ('0' ... '9')+
		\end{grammar}
		\caption{Modified TDL\textsuperscript{TP} grammar}\label{bnf:modified}
	\end{GrammarWrapper}

	\newpage

	\par We describe the labels used in grammar~\ref{bnf:modified} below: % https://en.wikibooks.org/wiki/LaTeX/Tables

	\begin{table}[h]
		\caption{Expression Types}
		\centering
		\makebox[\textwidth]{
			\begin{tabular}{l l l}
				\hline
				Label 	& Expression Type						& Concrete TDL example \\
				\hline
				UQTE	& Universally quantified trapset expr.		& \texttt{A(!TS1)} \\
				EQTE 	& Existentially quantified trapset expr.	& \texttt{E(TS1;TS2)} \\
				NE 		& Negated 									& \texttt{\texttilde A(TS1/TS2)} \\
				BE 		& Binary 									& \texttt{A(!TS1) \& (E(!TS1) | A(TS1/TS2))} \\
				LTE 	& Leads-to 									& \texttt{E(TS2;TS3)\texttilde\textgreater E(TS1;TS2)} \\
				BLTE 	& Time-bounded leads-to						& \texttt{E(TS2;TS3)\texttilde\textgreater [\textgreater 2] E(TS1;TS2)} \\
				CRE 	& Conditional repetition 					& \texttt{\#A(TS1/TS2)[\textgreater 3]} \\
				CTE 	& Absolute trapset complement				& \texttt{!TS} \\
				RCTE 	& Relative trapset complement				& \texttt{TS1/TS2} \\
				LPTE 	& Linked trapset pair 						& \texttt{TS1;TS2} \\
			\end{tabular}
		}
		\label{tbl:expr-types}
	\end{table}

	\bigskip

	\par Note that grammar~\ref{bnf:modified} contains several operators for which we should define precedence rules. These rules are listed below \textit{(need to adjust these before moving to thesis)}:

	\begin{table}[h]
		\caption{Operator precedence}
		\makebox[\textwidth]{
			\begin{tabular}{l l l l}
				\hline
				Type & Argument domain & Operator & Precedence \\
				\hline
				\multirow{3}{*}{Trapset expression} & \multirow{3}{*}{Trapset symbols}
				  & Absolute complement (\texttt{!}) & 0 \\
				& & Relative complement (\texttt{\textbackslash}) & 1 \\
				& & Linked pair 		(\texttt{;}) & 2 \\
				\hline
				\multirow{2}{*}{Trapset quantifier} & \multirow{2}{*}{Trapset expressions}
				  & Existential quantification (\texttt{E}) & 0 \\
				& & Universal quantification   (\texttt{A}) & 0 \\
				\hline
				\multirow{8}{*}{Logical} & \multirow{8}{*}{Trapset quantifiers}
				  & Negation (\texttt{\texttilde}) & 0 \\
				& & Conjunction (\texttt{\&}) & 1 \\
				& & Disjunction (\texttt{|}) & 2 \\
				& & Implication (\texttt{-\textgreater}) & 3 \\
				& & Equivalence (\texttt{\textless-\textgreater}) & 4 \\
				& & Leads to (\texttt{\texttilde\textgreater}) & 5 \\
				& & Time-bounded leads to (\texttt{\texttilde\textgreater [IntegerRelation]}) & 5 \\
				& & Conditional repetition (\texttt{\# ... [IntegerRelation]}) & 6 \\
			\end{tabular}
		}
		\label{tbl:oper-prec}
	\end{table}

	\newpage

	\subsection*{Requirements Summary}
	\par The TDL\textsuperscript{TP} parser needs to satisfy these primary requirements:
	\begin{itemize}
		\item able to parse a valid expression from grammar~\ref{bnf:modified} and report basic errors for invalid expressions;
		\item adheres to the precedence rules defined in table~\ref{tbl:oper-prec};
		\item provides facilities for the construction of a concrete syntax tree (parse tree). 
	\end{itemize}

	\section*{Implementation}
	\par The following subsections present information on the implementation of the TDL\textsuperscript{TP} parser.

	\subsection*{Technology}
	\par Because we are essentially implementing a proof-of-concept, our main criteria for choosing the appropriate technology are:
	\begin{itemize}
		\item ease-of-use (readability, gradual learning curve);
		\item support for multiple target platforms;
		\item support for parse tree generation.
	\end{itemize}.

	\par In order to save time, it makes sense to use an existing toolkit which satisfies the above-mentioned criteria.

	\bigskip

	\par ANTLR (ANother Tool for Language Recognition, \cite{antlrsite}) is our preferred parser generation tool. Given a grammar, it will generate a (top-down) parser which is capable of building parse trees. We chose ANTLR (version 4) because it supports the following features:
	\begin{itemize}
		\item multiple target languages (including \texttt{Java}, \texttt{Python});
		\item availability of testing and debugging utilities (e.g AST graph generation, lexer debugging);
		\item direct left recursion \cite{antlrrecursion};
		\item generation of parsers whose parsing strategy (\textit{ALL(*)}) tends to exhibit linear time complexity behavior in practice \cite{antlrcomplex}.
	\end{itemize}

	Terence Parr, the author of ANTLR 4, provides a comprehensive ANTLR 4 tutorial in \cite{antlrtutorial}.

	\subsection*{Technical Details}
	\par This section presents implementation details related to using ANTLR 4 to generate a parser for TDL\textsuperscript{TP}.

	\subsubsection*{ANTLR 4 grammar}
	\par ANTLR has its own grammar definition language (described in \cite{antlrtutorial}). The result of converting grammar~\ref{bnf:modified} to an ANTLR  grammar is displayed in the listing below:
	
	\begin{lstlisting}[
		basicstyle=\small,
		caption={ANTLR 4 grammar for TDL\textsuperscript{TP}},label={lst:antlr-gram}
	]
grammar TDLExpressionLanguage;

// Root production:
expression
: LEFT_PAREN expression RIGHT_PAREN  # GroupedExpression
| LOP_NEGATION expression  # NegatedExpression
| expression LOP_CONJUNCTION expression  # ConjunctiveExpression
| expression LOP_DISJUNCTION expression  # DisjunctiveExpression
| expression LOP_IMPLICATION expression  # ImplicativeExpression
| expression LOP_EQUIVALENCE expression  # EquivalenceExpression
| expression LOP_LEADS_TO expression  # LeadsToExpression
| expression LOP_LEADS_TO
	LEFT_BRACKET boundOverNaturals RIGHT_BRACKET expression 
	# TimeBoundedLeadsToExpression
| LOP_REPETITION_COUNT expression
	LEFT_BRACKET boundOverNaturals RIGHT_BRACKET 
	# ConditionalRepetitionExpression
| quantifiedTrapsetExpression  # GroundTermExpression
;

// Ground term for expression (quantified trapset expression):
quantifiedTrapsetExpression
: LOP_UNIVERSAL_QUANTIFIER
	LEFT_PAREN trapsetExpression RIGHT_PAREN
	# UniversalTrapsetExpression
| LOP_EXISTENTIAL_QUANTIFIER
	LEFT_PAREN trapsetExpression RIGHT_PAREN
	# ExistentialTrapsetExpression
;

// Trapset expressions:
trapsetExpression
: LEFT_PAREN trapsetExpression RIGHT_PAREN
	# GroupedTrapsetExpression
| TOP_ABSOLUTE_COMPLEMENT trapsetExpression
	# AbsoluteTrapsetComplementExpression
| trapsetExpression TOP_RELATIVE_COMPLEMENT trapsetExpression
	# RelativeTrapsetComplementExpression
| trapsetExpression TOP_LINKED_PAIR trapsetExpression
	# LinkedTrapsetPairExpression
| TRAPSET_ID
	# TrapsetIdentifierExpression
;

// Partial relation over natural numbers:
boundOverNaturals : REL_LESS_THAN NATURAL_NUMBER  # LessThanBound
| REL_GREATER_THAN NATURAL_NUMBER  # GreaterThanBound
| REL_LESS_THAN_OR_EQ NATURAL_NUMBER  # LessThanOrEqBound
| REL_GREATER_THAN_OR_EQ NATURAL_NUMBER  # GreaterThanOrEqBound
| REL_EQUAL NATURAL_NUMBER  # EqualityBound
;

// Tokens:
// Grouping tokens:
LEFT_BRACKET    : '[' ;
RIGHT_BRACKET   : ']' ;
LEFT_PAREN      : '(' ;
RIGHT_PAREN     : ')' ;

// Logical operators:
LOP_UNIVERSAL_QUANTIFIER    : [Aa] ;
LOP_EXISTENTIAL_QUANTIFIER  : [Ee] ;
LOP_NEGATION                : '~' ;
LOP_CONJUNCTION             : '&' ;
LOP_DISJUNCTION             : '|' ;
LOP_IMPLICATION             : '-' BLANK* '>' ;
LOP_EQUIVALENCE             : '<' BLANK* '-' BLANK* '>' ;
LOP_LEADS_TO                : '~' BLANK* '>' ;
LOP_REPETITION_COUNT        : '#' ;

// Trapset operators:
TOP_ABSOLUTE_COMPLEMENT : '!' ;
TOP_RELATIVE_COMPLEMENT : '\\' ;
TOP_LINKED_PAIR         : ';' ;

// Relations over natural numbers:
REL_LESS_THAN_OR_EQ     : '<' BLANK* '=' ;
REL_GREATER_THAN_OR_EQ  : '>' BLANK* '=' ;
REL_LESS_THAN           : '<' ;
REL_GREATER_THAN        : '>' ;
REL_EQUAL               : '=' ;

// Other:
TRAPSET_ID      : TRAPSET_PREFIX NUMERIC_ID? ;
NATURAL_NUMBER  : ZERO
| NONZERO_DIGIT ANY_DIGIT*
;
NUMERIC_ID      : ANY_DIGIT+ ;

// Fragments (only meant to be used in token productions):
fragment TRAPSET_PREFIX : [Tt][Rr] ;
fragment ZERO           : [0] ;
fragment ANY_DIGIT      : [0-9] ;
fragment NONZERO_DIGIT  : [1-9] ;
fragment BLANK          : [ \t\r\n] ;

// Meta rules:
// Skip whitespace:
WS : BLANK+ -> skip ;
	\end{lstlisting}

	\newpage

	\subsubsection*{Generated Parser Classes}
	\par When the grammar definition in listing~\ref{lst:antlr-gram} is processed by ANTLR, it will generate the following Java classes:
	\begin{itemize}
		\item \texttt{TDLExpressionLanguageLexer};
		\item \texttt{TDLExpressionLanguageParser};
		\item \texttt{TDLExpressionLanguageBaseListener};
		\item \texttt{TDLExpressionLanguageBaseVisitor}.
	\end{itemize}

	\par The final sections in this document are devoted to brief descriptions of the classes listed above.


	\subsubsection*{\texttt{TDLExpressionLanguageLexer}}
	\par Descendant of \texttt{org.antlr.v4.runtime.Lexer}. \\ A lexer in the ANTLR architecture is tasked with extracting token objects from an input character stream. Generally an instance of this class would be passed to an object whose class implements \texttt{org.antlr.v4.runtime.TokenStream}. The latter can then be used by the parser to extract grammatical structures from the token stream.

	\subsubsection*{\texttt{TDLExpressionLanguageParser}}
	\par Descendant of \texttt{org.antlr.v4.runtime.Parser}. \\ A parser in the ANTLR architecture is tasked with syntax recognition. It expects to receive a token stream when instantiated. A parser contains methods for generating a parse tree based on the token stream. This parse tree can either be visited by the corresponding \texttt{org.antlr.v4.runtime.tree.ParseTreeVisitor} subclass or handed to a \texttt{org.antlr.v4.runtime.tree.ParseTreeWalker} instance with the corresponding \texttt{org.antlr.v4.runtime.tree.ParseTreeListener} subclass.

	\subsubsection*{\texttt{TDLExpressionLanguageBaseListener}}
	\par Implements \texttt{org.antlr.v4.runtime.tree.ParseTreeListener}. \\ Intended to be used with a parse tree and an instance of \\ \texttt{org.antlr.v4.runtime.tree.ParseTreeWalker}. Contains \texttt{enter...()} and \texttt{exit...()} methods for each production in the source grammar (alternatives are grouped under the same method name unless the \texttt{\#} symbol is used to specify a name for each alternative). These enter-exit methods are called during a tree walk as the walker traverses nodes in the parse tree. \\ The main advantage of this class over its \\ \texttt{org.antlr.v4.runtime.tree.ParseTreeVisitor} alternative is that we can simply define what should happen as nodes are encountered during a parse tree walk without worrying about the details of traversal. That is also its main weakness - limited control.

	\subsubsection*{\texttt{TDLExpressionLanguageBaseVisitor}}
	\par Implements \texttt{org.antlr.v4.runtime.tree.ParseTreeVisitor}. \\ It is based on the Visitor pattern from \cite{patternbook}. Provides a \texttt{visit...()} method for each production in the source grammar (alternatives are grouped under the same method name unless the \texttt{\#} symbol is used to specify a name for each alternative). Traversal begins by calling the \texttt{visit...()} method on the root of a parse tree instance. The visitor is then responsible for controlling the traversal of the tree. This is the visitor's main advantage over the listener described in the previous section - we can write proactive instead of reactive code. Additionally, the visitor is capable of returning a value from each of its \texttt{visit..()} methods.

	\printbibliography[
		title=Sources
	]

\end{document}
